\documentclass[11pt]{exam}
\usepackage{graphicx}
\usepackage{anysize}
\usepackage{dsfont}
\usepackage{fullpage}
\usepackage{verbatim}
\usepackage{dsfont}
\usepackage{url,amsfonts, amsmath, amssymb, amsthm,color, enumerate, multicol}
\usepackage{placeins}
\usepackage{amsmath}
\usepackage{mathtools}

%% Ceil and Floor
\DeclarePairedDelimiter{\ceil}{\lceil}{\rceil}
\DeclarePairedDelimiter\floor{\lfloor}{\rfloor}

\printanswers
%\noprintanswers

\begin{document}

\title{ EECS 203: Discrete Mathematics \\
        Homework 8 (Winter 2017)}
\date{Due: March 9, 2017}
\author{Kevin Zheng}

\maketitle

\begin{questions}

%%%%%%%%%%%%%%%%%%%%%%%%%%%%%%%%%%%%%%%%%%%%%%%%%%%%%%%%%%%%%%%%%%%%%%%%%%%%%%%%%%%%
%%%%%%%%%%%%%%%%%%%%%%%%%%%%%%%%%%%%%%%%%%%%%%%%%%%%%%%%%%%%%%%%%%%%%%%%%%%%%%%%%%%%
{\bf \item[] Textbook Questions:}
\question[2] Section 7.1 Problem 30\\
(“What is the probability that a player ...”) 
\begin{solution}\\
$C(6, 5) * C(34, 1) = 6 * 34 = 204$ \\
Dividing by the total amount of tickets, $C(40, 6) = 3838380$, \\
it is $204 / 3,838,380 = 5.3 \times 10^{-5}$
\end{solution}
%%%%%%%%%%%%%%%%%%%%%%%%%%%%%%%%%%%%%%%%%%%%%%%%%%%%%%%%%%%%%%%%%%%%%%%%%%%%%%%%%%%%
%%%%%%%%%%%%%%%%%%%%%%%%%%%%%%%%%%%%%%%%%%%%%%%%%%%%%%%%%%%%%%%%%%%%%%%%%%%%%%%%%%%%
\question[4] Section 7.1 Problem 34\\
(“What is the probability that Bo, Colleen, Jeff ...”) 
\begin{solution}\\
(a) There are $50 * 49 * 48 * 47$ outcomes, and only one outcome where Bo, Colleen, Jeff, and Rohini win consecutively. Thus, $1 / (50 * 49 * 48 * 47) = 1/5527200$\\
(b) The outcome is $50 * 50 * 50 * 50$. Similarly, there is only 1 outcome, so it is $1 / (50 * 50 * 50 * 50) = 1/6250000$

\end{solution}
%%%%%%%%%%%%%%%%%%%%%%%%%%%%%%%%%%%%%%%%%%%%%%%%%%%%%%%%%%%%%%%%%%%%%%%%%%%%%%%%%%%%
%%%%%%%%%%%%%%%%%%%%%%%%%%%%%%%%%%%%%%%%%%%%%%%%%%%%%%%%%%%%%%%%%%%%%%%%%%%%%%%%%%%%
\question[2] Section 7.1 Problem 36\\
(“Which is most likely: rolling a total of 8 ...”) 
\begin{solution}\\
There are 5 ways to get a total of 8 when rolling 2 dice: (6,2), (5,3), (4,4), (3,5), (2,6) and 36 total possibilities, so the probability is 5/36. \\
For 3 dice, there are 21 ways: 
Starting with ... \\
6 - 1 way (6, 1, 1) \\
5 - 2 ways (5, 1, 2) and (5, 2, 1)\\
4 - 3 ways (4, 1, 3), (4, 3, 1) and (4, 2, 2)\\
3 - 4 ways (3, 1, 2), (3, 2, 1), (3, 3, 2), (3, 2, 3)\\
Continuing this trend, the answer is 1 + 2 + 3 + 4 + 5 + 6 = 21. Since there are 216 possible ways to throw 3 dice, 21/216 = 0.097. 5/36 = .138 which is greater, so it is better to roll two dice.

\end{solution}
%%%%%%%%%%%%%%%%%%%%%%%%%%%%%%%%%%%%%%%%%%%%%%%%%%%%%%%%%%%%%%%%%%%%%%%%%%%%%%%%%%%%
%%%%%%%%%%%%%%%%%%%%%%%%%%%%%%%%%%%%%%%%%%%%%%%%%%%%%%%%%%%%%%%%%%%%%%%%%%%%%%%%%%%%
\question[6] Section 7.2 Problem 8\\
(“What is the probability of these events when ...”) 
\begin{solution}\\
(b) 1 is no different than any other number, so if 2 was randomnized, it can either be before or after 2. Thus the probability is 1/2. \\
(c) If 1 must immediately precede 2, then we can consider (1, 2) to be a single object. Thus it is now the probability of selecting this (1, 2) object out of n - 1 objects: $\frac{(n - 1)!}{n!}$ \\
(d) 1/4, because 1/2 of the permutations have n before 1, and half of those have n-1 before 2.
\end{solution}
%%%%%%%%%%%%%%%%%%%%%%%%%%%%%%%%%%%%%%%%%%%%%%%%%%%%%%%%%%%%%%%%%%%%%%%%%%%%%%%%%%%%
%%%%%%%%%%%%%%%%%%%%%%%%%%%%%%%%%%%%%%%%%%%%%%%%%%%%%%%%%%%%%%%%%%%%%%%%%%%%%%%%%%%%
\question[2] Section 7.2 Problem 24\\
(“What is the conditional probability that exactly four ...”) 
\begin{solution}\\
THHHH \\
The T is given, while each heads has a 1/2 chance to be flipped. \\ $1 \cdot 1/2 \cdot 1/2 \cdot 1/2 \cdot 1/2 = 1/16$ 
\end{solution}
%%%%%%%%%%%%%%%%%%%%%%%%%%%%%%%%%%%%%%%%%%%%%%%%%%%%%%%%%%%%%%%%%%%%%%%%%%%%%%%%%%%%
%%%%%%%%%%%%%%%%%%%%%%%%%%%%%%%%%%%%%%%%%%%%%%%%%%%%%%%%%%%%%%%%%%%%%%%%%%%%%%%%%%%%
\question[2] Section 7.2 Problem 26\\
(“Let E be the events that a randomly generated ...”) 
\begin{solution}\\
P(E) is 1/2 because there are $2^3 = 8$ total possibilities and 4 out of those possibilities have an odd number of 1's (100, 010, 001, 111) \\
P(F) is 1/2 because the first digit can either be 1 or 0. \\
$P(E \cap F)$ is 1/4 because there are 2/8 possibilities: 100 or 111. Thus, it's independent because $1/2 * 1/2 = 1/4$
\end{solution}
%%%%%%%%%%%%%%%%%%%%%%%%%%%%%%%%%%%%%%%%%%%%%%%%%%%%%%%%%%%%%%%%%%%%%%%%%%%%%%%%%%%%
%%%%%%%%%%%%%%%%%%%%%%%%%%%%%%%%%%%%%%%%%%%%%%%%%%%%%%%%%%%%%%%%%%%%%%%%%%%%%%%%%%%%
\question[4] Section 7.2 Problem 34\\
(“For each of the following probabilities when n ...”) 
\begin{solution}\\
Using the formula $P(x) = \frac{n!}{x!(n-x)!} p^x(1-p)^{n-x}$ \\
(a) $P(0) = \frac{n!}{0!(n-0)!}p^0(1-p)^{n}$ \\
= $\frac{n!}{n!}(1-p)^n = (1-p)^n$ \\
(b) When there is at least 1 success, the only case where that is false is when there are exactly 0 successes. Thus, 1 - (a) = $1 - (1-p)^n$ \\
(c) When there is at most 1 success, there are 2 cases: zero successes or 1 success. Thus, it's (a) + P(1). \\
P(1) = $\frac{n!}{1!(n-1)!}p^1(1-p)^{n - 1}$ \\
= $\frac{n!}{(n-1)!}p(1-p)^{n - 1}$ \\
= $np(1 - p)^{n - 1}$ \\
Thus it is (a) + $np(1 - p)^{n - 1}$ \\
= $(1 - p)^n + np(1 - p)^{n - 1}$ \\
(d) Similar to (b), this is just 1 - (c). $1 - (1 - p)^n + np(1 - p)^{n - 1}$
\end{solution}
%%%%%%%%%%%%%%%%%%%%%%%%%%%%%%%%%%%%%%%%%%%%%%%%%%%%%%%%%%%%%%%%%%%%%%%%%%%%%%%%%%%%
%%%%%%%%%%%%%%%%%%%%%%%%%%%%%%%%%%%%%%%%%%%%%%%%%%%%%%%%%%%%%%%%%%%%%%%%%%%%%%%%%%%%
\question[2] Section 7.3 Problem 6\\
(“When a test for steroids is given to soccer ...”) 
\begin{solution}\\
Given that U = Uses steroies and P = Tests positive: \\
$P(U | P) = \frac{P(P|U) P(U)}{P(P|U)P(U) + P(P|\bar{U})P(\bar{U})}$ \\
= $\frac{.98 * .05}{.98 * .05 + .02 * .95}$ \\
= $0.3$
\end{solution}
%%%%%%%%%%%%%%%%%%%%%%%%%%%%%%%%%%%%%%%%%%%%%%%%%%%%%%%%%%%%%%%%%%%%%%%%%%%%%%%%%%%%
%%%%%%%%%%%%%%%%%%%%%%%%%%%%%%%%%%%%%%%%%%%%%%%%%%%%%%%%%%%%%%%%%%%%%%%%%%%%%%%%%%%%
\question[6] Section 7.3 Problem 10\\
(“Suppose that 4 of the patients tested ...”) 
\begin{solution}\\
Given that I = Infected with avian influenza and P = Tests positvie: \\
(a) $P(I | P) = \frac{P(P|I) P(I)}{P(P|I)P(I) + P(P|\bar{I})P(\bar{I})}$ \\
= $\frac{.97 * .04}{.97 * .04 + .02 * .96} = .669$ \\
(b) This is the complement, so it is $1 - .669 = .331$ \\
(c) $P(I | \bar{P}) = \frac{P(\bar{P}|I) P(I)}{P(\bar{P}|I)P(I) + P(\bar{P}|\bar{I})P(\bar{I})}$ \\
= $\frac{.03 * .04}{.03 * .04 + .98 * .96} = .001$ \\
(d) This is the complement, so it is $1 - .001 = .999$
\end{solution}
%%%%%%%%%%%%%%%%%%%%%%%%%%%%%%%%%%%%%%%%%%%%%%%%%%%%%%%%%%%%%%%%%%%%%%%%%%%%%%%%%%%%
%%%%%%%%%%%%%%%%%%%%%%%%%%%%%%%%%%%%%%%%%%%%%%%%%%%%%%%%%%%%%%%%%%%%%%%%%%%%%%%%%%%%
\question[4] Section 7.3 Problem 16\\
(“Ramesh can get to work in three different ...”) 
\begin{solution}\\
Given: L = late, C = car, B = Bike, U = B[U]s. \\
(a)$P(C | L) = \frac{P(L|C)P(C)}{P(L|C)P(C) + P(L|B)P(B) + P(L|U)P(U)}$ \\
We can cancel out the P(C), P(B), and P(U) because they are all 1/3. \\
= $\frac{.5}{.5 + .05 + .2} = 2/3$ \\
(b) By simply changing the values, \\
$P(C | L) = \frac{.5 * .3}{.5*.3 + .05*.6 + .2*.1} = .75$

\end{solution}
%%%%%%%%%%%%%%%%%%%%%%%%%%%%%%%%%%%%%%%%%%%%%%%%%%%%%%%%%%%%%%%%%%%%%%%%%%%%%%%%%%%%
%%%%%%%%%%%%%%%%%%%%%%%%%%%%%%%%%%%%%%%%%%%%%%%%%%%%%%%%%%%%%%%%%%%%%%%%%%%%%%%%%%%%
\end{questions}


\end{document}